\documentclass{article}
\usepackage[utf8]{inputenc}
\usepackage[spanish]{babel}
\usepackage{amsmath}
\usepackage{tabularx}

\title{Prestaciones}
\author{David Mato}
\date{} % Para eliminar la fecha

\begin{document}
	
	\maketitle % Genera el título
	
	\section*{Nacimiento y cuidado de un menor}
	
	\subsection*{Situaciones protegidas}
	
	Se consideran situaciones protegidas, durante los períodos de descanso y permisos que se disfruten por tales situaciones:
	
	\begin{itemize}
		\item El nacimiento de hijo o hija.
		\item La adopción, la guarda con fines de adopción y el acogimiento familiar, y se trate de:
		\begin{itemize}
			\item Menores de 6 años.
		\end{itemize}
	\end{itemize}
	
	\subsection*{Duración}
	
	Duración total para cada progenitor: 16 semanas (ampliable según supuestos)
	
	\begin{itemize}
		\item 6 obligatorias, ininterrumpidas y a jornada completa, inmediatamente posteriores a la fecha del parto, de la resolución judicial o decisión administrativa.
		\item Las 10 semanas restantes, en periodos semanales, dentro de los 12 meses siguientes al parto.
	\end{itemize}
	
	\subsection*{Prestación económica}
	
	Al 100 \% de la B.R.
	
	En caso de parto múltiple y de adopción o acogimiento de más de un menor, se concederá un subsidio especial por cada hijo, igual al que corresponda percibir por el primero, durante el período de 6 semanas.
	
	\bigskip
	
	\hrule
	
	\section*{Subsidios}
	
	\subsection*{Plazo de solicitud}
	
	Se amplía a 6 meses desde la fecha del hecho causante del subsidio, pudiendo solicitarse en cualquier momento dentro de esos 6 meses. \textbf{El derecho nace el mismo día de la solicitud. Desaparece el concepto de días consumidos.}
	
	\subsection*{Mes de espera}
	
	\textbf{Se elimina el plazo del mes de espera} de un mes tras la fecha de agotamiento de la prestación contributiva.
	
	\subsection*{Parcialidad}
	
	\textbf{Se elimina la parcialidad de los subsidios} de cotización insuficiente y agotamiento de prestación contributiva.
	
	\subsection*{Días consumidos}
	
	\textbf{Desaparece el concepto de días consumidos. El nacimiento del derecho será el día de la solicitud con carácter general.}
	
	\subsection*{Cuantía nuevos subsidios}
	
	\begin{tabular}{|l|l|l|}
		\hline
		6 primeros meses & 6 meses siguientes & Resto del período \\
		\hline
		95 \% del IPREM & 90 \% del IPREM & 80 \% del IPREM \\
		\hline
		570 € & 540 € & 480 € \\
		\hline
	\end{tabular}
	
	\subsection*{Cuantía del subsidio para mayores de 52 años}
	
	Cuantía fija del 80 \% del IPREM.
	
	\subsection*{Reconocimiento trimestral}
	
	Se reduce a \textbf{un trimestre el periodo de reconocimiento. Es decir, el subsidio se aprueba por un período trimestral tras el que hay que realizar una solicitud de prórroga del subsidio.}
	
	\subsection*{Declaración responsable}
	
	Se exigirá una \textbf{declaración responsable de rentas e ingresos al solicitante} y la obligación de presentar anualmente la \textbf{declaración del IRPF}.
	
	\subsection*{Requisito de rentas}
	
	\textbf{Se suprime el doble requisito de carencia de rentas} propias y de la unidad familiar.
	
	\subsection*{Complemento de apoyo al empleo}
	
	Los perceptores de subsidio podrán compatibilizar el subsidio temporalmente hasta un máximo de 180 días.
	
	\subsection*{Compatibilidad con becas y ayudas}
	
	Se establece la \textbf{compatibilidad de las prestaciones por desempleo con las becas y ayudas} que se obtengan por asistencia a acciones de formación profesional o en el trabajo.
	
	\subsection*{Tipos subsidios}
	
	\begin{itemize}
		\item De agotamiento prestación contributiva.
		\item De cotizaciones insuficientes.
		\item De emigrantes retornados.
		\item De víctimas de violencia de género y sexual.
		\item +52 años.
	\end{itemize}
	
	\bigskip
	
	\hrule
	
	\section*{Subsidio desempleo por agotamiento prestación contributiva}
	
	\subsection*{Requisitos}
	
	\begin{itemize}
		\item Estar en desempleo total.
		\item Haber agotado el día 1 de noviembre de 2024 o con posterioridad una prestación contributiva por desempleo.
		\item No tener derecho a la prestación contributiva por desempleo.
		\item Estar inscrito o inscrita como demandante de empleo en el momento de la resolución de la solicitud.
	\end{itemize}
	
	\subsection*{Duración}
	
	La duración máxima del subsidio por desempleo se determinará en función de:
	
	\begin{itemize}
		\item la edad que tengas en la fecha de agotamiento de la prestación por desempleo.
		\item la acreditación de responsabilidades familiares.
		\item la duración de la prestación por desempleo agotada con arreglo a la siguiente tabla:
	\end{itemize}
	
	\begin{tabular}{|l|l|l|}
		\hline
		Edad en la fecha de agotamiento de la prestación & Duración de la prestación por desempleo agotada & Duración máxima del subsidio \\
		\hline
		$<$45 & $>=$360 días & 6 meses \\
		\hline
		$>$45 & $>=$120 días & 6 meses \\
		\hline
	\end{tabular}
	
	\bigskip
	
	\hrule
	
	\section*{Subsidio por cotización insuficiente}
	
	\subsection*{Requisitos}
	
	\begin{itemize}
		\item Haber trabajado al menos 90 días.
		\item Estar en situación legal de desempleo a partir del 1 de noviembre de 2024.
	\end{itemize}
	
	\subsection*{Duración}
	
	La duración máxima del subsidio por desempleo se determinará en función de:
	
	\begin{itemize}
		\item el periodo de ocupación cotizado.
		\item la acreditación de responsabilidades familiares.
	\end{itemize}
	
	Con arreglo a la siguiente tabla:
	
	\begin{tabular}{|l|l|l|l|}
		\hline
		Periodo mínimo de ocupación cotizada & Acreditación de responsabilidades familiares & Duración máxima del subsidio \\
		\hline
		90 días & No & 3 meses \\
		\hline
		120 días & No & 4 meses \\
		\hline
		150 días & No & 5 meses \\
		\hline
		180 días & Si & 21 meses \\
		\hline
	\end{tabular}
	
	Las cotizaciones que sirvieron para el nacimiento de este subsidio no se tendrán en cuenta para el reconocimiento en el futuro de la prestación contributiva o del subsidio por desempleo.
	
	\bigskip
	
	\hrule
	
	\section*{Subsidio mayores de 52 años}
	
	\subsection*{Requisitos}
	
	\begin{itemize}
		\item Tener 52 años en la fecha en la que se encuentre.
		\item Cumplir todos los requisitos, salvo la edad, para acceder a cualquier tipo de pensión contributiva de jubilación en el sistema de la Seguridad Social.
		\item Haber cotizado efectivamente en España por la contingencia de desempleo durante, al menos, seis años a lo largo de tu vida laboral.
	\end{itemize}
	
	\subsection*{Cuantía}
	
	No obstante, dichas cotizaciones sí tendrán efecto:
	
	\begin{itemize}
		\item Para el cálculo de la base reguladora de la pensión de jubilación y porcentaje aplicable a aquella.
		\item Para completar el tiempo necesario para el acceso a la jubilación anticipada.
	\end{itemize}
	
	\bigskip
	
	\hrule
	
	\section*{Jubilación ordinaria}
	
	\subsection*{Requisitos generales}
	
	\begin{itemize}
		\item Las personas incluidas en el Régimen general, afiliadas y en alta o en situación asimilada a la de alta, que reúnan las condiciones de edad, período mínimo de cotización y hecho causante, legalmente establecidos.
		\item También serán beneficiarios los trabajadores afiliados al Sistema de la Seguridad Social que, en la fecha del hecho causante, no estén en alta o en situación asimilada al alta, siempre que reúnan los requisitos de edad y cotización establecidos.
	\end{itemize}
	
	\subsection*{Determinación del período cotizado}
	
	\begin{itemize}
		\item Para el cómputo de los años y meses de cotización se tomarán años y meses completos, sin que se equiparen a un año o a un mes las fracciones de los mismos.
	\end{itemize}
	
	\subsection*{Base reguladora}
	
	A partir del año 2022,
	
	B.R.= Bases de cotización de los 300 meses inmediatamente anteriores/350
	
	Para aquellas personas a las que les sea aplicable la legislación anterior a 1-1-2013,
	
	B.R.= Bases de cotización de los 180 meses inmediatamente anteriores/210
	
	\subsection*{Complementos trabajadores con edad superior a la legalmente establecida}
	
	Cuando se acceda a la pensión de jubilación a una edad superior a la edad ordinaria de jubilación aplicable en cada caso, siempre que al cumplir esta edad se hubiera reunido el período mínimo de cotización exigido, se reconocerá al interesado por cada año completo cotizado desde que reunió los requisitos para acceder a esta pensión.
	
	\begin{enumerate}
		\item Un porcentaje adicional del 4 \% por cada año completo cotizado entre la fecha en que cumplió dicha edad y la del hecho causante de la pensión.
	\end{enumerate}
	
\end{document}