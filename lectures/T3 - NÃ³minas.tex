\documentclass{article}
\usepackage[utf8]{inputenc}
\usepackage[spanish]{babel}
\usepackage{amsmath}
\usepackage{amssymb} % Añadido el paquete amssymb
\usepackage{tabularx}

\title{La Nómina}
\author{David Mato}
\date{} % Para eliminar la fecha

\begin{document}
	
	\maketitle % Genera el título
	
	\textbf{Es un recibo en el que la empresa acredita el pago de las diferentes cantidades de dinero que conforman el sueldo de un trabajador.}
	
	El pago ha de realizarse documentalmente, de manera que el empresario deberá entregar al trabajador un recibo individual y justificativo del pago del mismo.
	
	El modelo más generalizado de nómina está constituida por tres grandes bloques: encabezamiento, devengos y deducciones.
	
	\section*{Estructura de la Nómina}
	
	\begin{itemize}
		\item \textbf{Período de Liquidación:} Recoge los días trabajados del mes al que corresponde la liquidación, incluyendo los festivos. Si el trabajador pertenece a un grupo de cotización de retribución mensual, siempre el total de días será 30, aunque el mes sea de 28, 29 o 30.
		\item \textbf{Devengos:} Existen dos tipos de devengos:
		\begin{itemize}
			\item \textbf{Percepciones salariales:} cotizan a la Seguridad Social
			\begin{itemize}
				\item Salario base: no lo establece la empresa, sino que cada categoría profesional tiene fijado uno en cada convenio colectivo o, en su ausencia, en las tablas de la Seguridad Social.
				\item Complementos salariales: personales, por puesto de trabajo …
			\end{itemize}
			\item \textbf{Percepciones no salariales:} (no cotizan a la Seguridad Social). Las más habituales son las dietas de viaje, los gastos de locomoción, los pluses de distancia y las indemnizaciones por traslado a otro centro de trabajo.
			
			Las percepciones no salariales no se tienen en cuenta para calcular la base de cotización del trabajador, es decir, no cotizan, salvo que excedan de los límites establecidos.
		\end{itemize}
	\end{itemize}
	
	\section*{El Salario}
	
	\subsection*{Salario Mínimo Interprofesional (SMI)}
	
	\textbf{¿Qué es el Salario Mínimo Interprofesional?}
	
	Es la cuantía retributiva mínima que percibe el trabajador referido a la jornada legal de trabajo. Lo regula el Gobierno anualmente, previa consulta con las organizaciones sindicales y las asociaciones empresariales más representativas. En el salario mínimo se computa únicamente la retribución en dinero.
	
	\begin{itemize}
		\item El Salario Mínimo Interprofesional se refiere al salario bruto de los trabajadores.
		\item El incremento aprobado en Consejo de Ministros se aplicará con carácter retroactivo.
	\end{itemize}
	
	\subsection*{Salario Base}
	
	El salario base o sueldo base es la \textbf{retribución económica fija que recibe el trabajador de acuerdo al tiempo o trabajo ejecutado}. Es la cantidad mínima, en la cual no se incluyen los \underline{complementos salariales}.
	
	En ocasiones se confunde el salario base con el salario mínimo interprofesional. Este último es la cantidad mínima de dinero que recibe un empleado por su trabajo, teniendo en cuenta todos los conceptos de la nómina, como los complementos salariales.
	
	\textbf{¿Qué diferencia hay entre salario, salario base y SMI?}
	
	Aunque son conceptos relacionados, no significan lo mismo:
	
	\begin{itemize}
		\item El \textbf{salario} es la retribución en dinero o especie que recibe un trabajador por su trabajo. Es el total bruto que se recibe en la \underline{nómina}.
		\item El \underline{salario base} es un concepto fijo de la nómina que no se puede modificar y se establece según se prevea en el \textbf{convenio colectivo} de aplicación. No se pacta, sino que viene determinado en las tablas salariales del convenio, en función de la categoría profesional del empleado. A diferencia del SMI, el salario base no lo establece el Gobierno, sino que se concreta en la negociación del convenio colectivo al que pertenece cada empresa.
		\item El \underline{salario mínimo interprofesional} (SMI) es la remuneración total mínima que una empresa paga al trabajador. Se trata de una cantidad fijada anualmente por el Gobierno, previa consulta con las organizaciones sindicales y asociaciones empresariales más representativas.
	\end{itemize}
	
	\subsection*{Salario en Especie}
	
	Hay ocasiones en que las empresas entregan a sus empleados bienes y servicios para su uso particular, bien de forma gratuita, bien a un precio inferior al del mercado, y lo hacen en retribución por el trabajo realizado. Estos bienes y servicios son conocidos como \textbf{salario en especie o retribución flexible}, forman parte del salario total y, por tanto, se contabilizan en la \underline{nómina}.
	
	El salario en especie \textbf{no puede representar más del 30\% del salario total del trabajador}, según establece el Estatuto de los Trabajadores. Además el cobro de una parte del salario en especie tampoco puede dar lugar a la minoración de la cuantía íntegra en dinero del salario mínimo interprofesional, es decir, que \textbf{quienes cobran el SMI deben tener garantizada la percepción en dinero de esa cantidad}, y el salario en especie será a mayores.
	
	\textbf{Ejemplos más comunes:} uso de vivienda, uso vehículo empresa, servicio gratuito de guardería, cestas de Navidad …
	
	\subsection*{IPREM}
	
	El \textbf{Indicador Público de Renta de Efectos Múltiples, IPREM}, es un índice cuyo \underline{valor} se establece en los Presupuestos Generales del \underline{Estado} y \textbf{establece el límite de las rentas personales y familiares que sirve de base para obtener determinadas ayudas sociales}: becas, ayudas escolares, asistencia jurídica gratuita, subvenciones o subsidio de \underline{paro} …
	
	El IPREM se compara con los ingresos brutos del interesado.
	
	\section*{Horas Complementarias}
	
	Las \textbf{horas complementarias} son aquellas que exceden de la jornada laboral establecida en un contrato a jornada parcial.
	
	\begin{itemize}
		\item Nunca podrán superar las 40 horas semanales de un trabajador contratado a jornada completa.
		\item El trabajador/a deberá conocer el día y la hora de realización de las horas pactadas con un preaviso mínimo de tres días.
		\item Se remuneras igual que las horas ordinarias.
	\end{itemize}
	
	\textbf{Tipos de horas complementarias:}
	
	\begin{itemize}
		\item \textbf{Horas complementarias pactadas:}
		\begin{itemize}
			\item Deben reflejarse por escrito y anexarse al contrato
			\item El contrato a jornada parcial no puede ser inferior a horas semanales en cómputo anual.
			\item No puede exceder del 30\% de las horas ordinarias establecidas por contrato.
		\end{itemize}
		\item \textbf{Horas complementarias voluntarias:}
		\begin{itemize}
			\item En personal con contrato indefinido y a tiempo parcial.
			\item Negarse a su realización no supone sanción.
			\item No puede exceder del 15\% de las horas ordinarias establecidas por contrato.
		\end{itemize}
	\end{itemize}
	
	\section*{Horas Extraordinarias}
	
	Tendrán consideración de horas extraordinarias cada hora de trabajo que se realice sobre la duración máxima de la jornada ordinaria.
	
	\textbf{Tipos de horas extraordinarias:}
	
	\begin{itemize}
		\item \checkmark \textbf{Horas extraordinarias fuerza mayor (Estructurales):} Son las que vengan exigidas por la necesidad de prevenir o reparar siniestros u otros daños extraordinarios y urgentes, como el riesgo de pérdida de materias primas. Es obligatoria su realización por el trabajador. No se tendrán en cuenta para el límite máximo anual de horas extraordinarias. Se compensarán como horas extraordinarias.
		\item \checkmark \textbf{Horas extraordinarias (No estructurales):} Se abonarán económicamente o se compensarán con descanso por pacto individual o colectivo. La cuantía a percibir por cada hora extraordinaria en ningún caso podrá ser inferior al valor de la hora ordinaria o se compensarán por tiempos equivalentes de descanso retribuido. En ausencia de pacto al respecto, se entenderá que las horas extraordinarias realizadas deberán ser compensadas mediante descanso dentro de los cuatro meses siguientes a su realización. El número máximo de horas extraordinarias al año a realizar por un trabajador será de 80. No se computarán a estos efectos las que hayan sido compensadas mediante descanso dentro de los cuatro meses siguientes a su realización.
	\end{itemize}
	
	\section*{Período de Prueba}
	
	Su establecimiento es optativo y de acordarlo deberán fijarlo por escrito en el contrato. Su duración máxima se establecerá en los convenios colectivos y en su defecto la duración no podrá exceder de:
	
	\begin{itemize}
		\item Seis meses para los técnicos titulados.
		\item Dos meses para el resto de los trabajadores.
	\end{itemize}
	
	En las empresas con menos de 25 trabajadores, el período de prueba no podrá exceder de tres meses para los trabajadores que no sean técnicos titulados.
	
	El período de prueba se computa a efectos de antigüedad.
	
	\section*{Vacaciones}
	
	\textbf{Duración:}
	
	\begin{itemize}
		\item Pactada en forma individual o colectiva. Nunca inferior a treinta días naturales.
		\item Se fijará de acuerdo entre empresario y trabajador de conformidad con lo establecido en su caso en los convenios colectivos sobre planificación anual de las vacaciones.
		\item El calendario de vacaciones se fijará en cada empresa. El trabajador conocerá las fechas que le corresponden dos meses antes, al menos, del comienzo previsto para las mismas.
	\end{itemize}
	
	Las vacaciones no son sustituibles por compensación económica, salvo en caso de extinción de contrato de trabajo que imposibilite el disfrute de las mismas.
	
	\section*{Anticipo de Salario}
	
	El trabajador tendrán derecho a percibir, sin que llegue el día señalado para el pago, anticipos a cuenta del trabajo ya realizado. Los anticipos que las empresas pueden hacer a los trabajadores sobre los salarios a percibir en el futuro carecen de regulación legal y se regirán, de concederse, por lo acordado entre las partes o la regulación existente en el convenio colectivo de aplicación.
	
	\textbf{Tipos de anticipo:}
	
	\begin{itemize}
		\item \checkmark Anticipo nómina por trabajo ya realizado (obligatorio).
		\item \checkmark Anticipo nómina por trabajo a realizar (no obligatorio).
		\item \checkmark Anticipo nómina por horas extra (no obligatorio).
	\end{itemize}
	
	\textbf{Características anticipo por trabajo ya realizado:}
	
	\begin{itemize}
		\item Obligatorio para todas las empresas.
		\item Una vez solicitado, el empresario debe responder con la máxima celeridad.
		\item Sólo se podrá solicitar el importe proporcional que haya generado en los días que haya trabajado.
		\item NO se establece límite pero la naturaleza del préstamo es puntual.
	\end{itemize}
	
	\section*{Préstamo Empresa a Sus Trabajadores}
	
	\textbf{Un empleado puede solicitar un préstamo a la empresa, o bien porque el \underline{convenio colectivo} prevé esta posibilidad, o bien de forma individual siempre que exista un acuerdo para ello}. En caso de que el préstamo privado otorgado por la empresa, suponga una ventaja respecto los tipos de interés habituales en el mercado, esta "mejora" adquirirá consideración de \underline{salario en especie}.
	
	\textbf{¿Cuándo existirá retribución en especie?}
	
	El tipo de interés que se aplicará al préstamo concedido será el porcentaje de interés legal del dinero para cada año. En los supuestos en que se pacte que el préstamo se devolverá sin interés, o con un porcentaje de interés distinto (inferior) al interés legal del dinero, dicha cuantía será considerada \underline{retribución en especie}.
	
	\section*{Embargo de Nómina}
	
	El \textbf{embargo de la nómina} supone la \textbf{retención por parte del empresario de una parte de la nómina de su empleado para proceder al pago de una deuda de dicho trabajador}.
	
	A diferencia del embargo de cuenta corriente en el que se embarga el dinero que tiene la persona en su cuenta bancaria, en el \textbf{embargo de nómina} el dinero que se retiene no llega a la cuenta del trabajador.
	
	Es inembargable el salario, sueldo, pensión, retribución o su equivalente, que no exceda de la cuantía señalada para el salario mínimo interprofesional.
	
	\textbf{Solo un juzgado o una administración pública pueden ordenar el embargo de una nómina.}
	
	El único caso en el que se permite \textbf{el embargo de la nómina por debajo del SMI se da cuando se trata de impago de pensión de alimentos}. En este supuesto es el Juez el que determina la parte del salario que se embargará.
	
	\section*{Mecanismo de Equidad Intergeneracional (MEI)}
	
	El Mecanismo de Equidad Intergeneracional (MEI) es un nuevo concepto de cotización que ha entrado en vigor en enero de 2023.
	
	Tiene como objetivo aumentar los ingresos de la Seguridad Social para hacer frente al pago de las pensiones y viene a sustituir al factor de sostenibilidad.
	
	Sobre esta cotización no se podrá aplicar ningún tipo de reducción o bonificación.
	
	Se aplicará en todos los casos en los que se cotice para la jubilación, se esté en \underline{situación de alta} o en situación asimilada al alta. Por tanto, como el \textbf{desempleo} es una situación asimilada al alta y cuando percibimos la prestación por desempleo, cotizamos para la jubilación, también se nos aplicará en estos casos.
	
	No obstante, esta nueva cotización no les será descontada a \textbf{las y los pensionistas jubilados}, puesto que no cotizan ya por jubilación.
	
	\section*{RLT y RNT}
	
	El \textbf{RLC y RNT o Recibo de Liquidación de Cotizaciones y Relación Nominal de Trabajadores} son archivos que vienen a sustituir los \textbf{antiguos modelos TC1 y TC2} que eran y son utilizados en el método de transmisión RED Internet de la Seguridad social.
	
	\subsection*{Relación Nominal de Trabajadores}
	
	La \textbf{Relación Nominal de Trabajadores} es el documento que \textbf{incluye la información de cotización de los trabajadores en un periodo de liquidación}
	
	EL RNT (Relación Nominal de Trabajadores) es un documento oficial expedido por la Tesorería General de la Seguridad Social a las empresas en el que se confirma la relación de empleados dados de alta en un periodo concreto (establecido mensualmente) y los devengos, bonificaciones o las deducciones generadas.
	
	\subsection*{Recibo de Liquidación de Cotizaciones}
	
	\textbf{El Recibo de Liquidación de Cotizaciones viene a ser lo mismo que el TC1 del método de transmisión RED Internet}, ya en desuso. Se trata del \textbf{documento de pago de las liquidaciones}, es decir, será el documento de ingreso que se detallarán las bases y los importes de cada uno de los trabajadores obteniendo el líquido total del envío.
	
	\textbf{¿Para qué sirve el RLC?}
	
	El modelo RLC (Recibo de Liquidación de Cotizaciones) es el documento oficial de pago de los seguros sociales, en dicho modelo se reflejará la información de cotización el importe correspondiente a las cotizaciones de los trabajadores.
	
\end{document}