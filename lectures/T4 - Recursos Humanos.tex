\documentclass{article}
\usepackage[utf8]{inputenc}

\title{El departamento de recursos humanos}
\author{David Mato}
\date{} % Para eliminar la fecha

\begin{document}
	
	\maketitle % Genera el título
	
	\date{} % Opcional: para omitir la fecha
	
	\maketitle
	
	Los recursos humanos son el conjunto de trabajadores o empleados que componen una empresa.
	
	Para llevar a cabo sus funciones, las empresas necesitan recursos, que en economía se denominan factores de producción. De todos los recursos, las personas son el recurso más importante. Sin personas es imposible que las empresas funcionen.
	
	\section{1- Funciones y estructura del departamento de recursos humanos}
	
	Las funciones atribuidas al departamento de recursos humanos no tienen la misma importancia en todas las empresas, pues dependen de múltiples factores como el tamaño de la empresa o la existencia de uno o varios centros de trabajo. No es lo mismo una empresa con 5 trabajadores que una con 1500.
	
	Las funciones asignadas al departamento de recursos humanos son las siguientes:
	
	\subsection*{Dirección estratégica}
	
	\begin{itemize}
		\item Diseña, planifica y ejecuta las estrategias de recursos humanos.
		\item Coopera con otros departamentos en la elaboración de las políticas de la empresa.
		\item Coordina el área de recursos humanos con el resto de áreas de la empresa.
	\end{itemize}
	
	\subsection*{Planificación, reclutamiento y selección}
	
	Su objetivo es proporcionar a las empresas el personal necesario en cada momento.
	
	Esta función incluye actividades como las siguientes:
	
	\begin{itemize}
		\item Realización de los perfiles profesionales de los puestos de trabajo.
		\item Reclutamiento y selección de nuevos trabajadores.
	\end{itemize}
	
	\subsection*{Relaciones laborales}
	
	Es la función encargada de garantizar la correcta relación entre empleadores y trabajadores.
	
	Dentro de esta función se encuentran actividades como:
	
	\begin{itemize}
		\item La negociación de convenios colectivos.
		\item Relaciones con los representantes de los trabajadores.
	\end{itemize}
	
	\subsection*{Administración de personal}
	
	Se ocupa de los procedimientos administrativos desde que una persona se incorpora a la empresa hasta que deja de formar parte de ella. Cabe destacar las siguientes actividades:
	
	\begin{itemize}
		\item Realización, firma y registro de contratos de trabajo.
		\item Elaboración de recibos de salarios.
	\end{itemize}
	
	\subsection*{Seguridad y salud laboral}
	
	La seguridad se refiere al conjunto de medidas técnicas, formativas, médicas y psicológicas para prevenir accidentes de trabajo.
	
	La salud se refiere al diagnóstico y prevención de enfermedades laborales.
	
	\subsection*{Desarrollo de recursos humanos}
	
	Se refiere a la necesidad de que los empleados estén motivados para trabajar más y mejor, y para desempeñar puestos de trabajo de mayor responsabilidad, con el objetivo de aumentar su productividad.
	
	\subsection*{Responsabilidad social de la empresa}
	
	La responsabilidad social se considera una de las políticas estratégicas de la empresa. A través de esta política estratégica las empresas adquieren compromisos con sus empleados y con su entorno social, más allá del beneficio inmediato, con el objetivo de mejorar la situación competitiva y su valor añadido.
	
\end{document}