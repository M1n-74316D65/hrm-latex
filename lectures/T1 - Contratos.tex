\documentclass{article}
\usepackage[utf8]{inputenc}
\usepackage[spanish]{babel}
\usepackage{amsmath}
\usepackage{amssymb}
\usepackage{longtable}

\title{Contratos de Trabajo}
\author{David Mato}
\date{} % Para eliminar la fecha

\begin{document}
	
	\maketitle % Genera el título
	
	\section*{Principios para la aplicación de las normas laborales}
	
	\begin{longtable}{|p{0.3\textwidth}|p{0.65\textwidth}|}
		\hline
		\textbf{Principios} & \textbf{Contenidos} \\
		\hline
		\endhead
		\textbf{Principio de norma mínima} & El principio de norma mínima significa que las normas de rango superior establecen el contenido mínimo de la norma que la sigue. Por lo tanto, las normas laborales de inferior rango pueden establecer mejores condiciones de trabajo que la norma de superior rango, pero nunca empeorarlas. \\
		\hline
		\textbf{Principio de norma más favorable} & 
		\begin{itemize}
			\item Cuando existan dos o más normas, cualquiera que sea su rango, aplicables a un caso concreto, se aplicará la que, apreciada en su conjunto, sea más favorable para el trabajador.
			\item La norma en cuestión se aplicará en su totalidad, es decir, no se puede tomar la favorable de una norma y rechazar lo adverso.
		\end{itemize} \\
		\hline
		\textbf{Principio de irrenunciabilidad de derechos} & Los trabajadores no pueden renunciar a los derechos que tengan reconocidos. \\
		\hline
		\textbf{Principio de condición más beneficiosa} & Si una norma laboral establece condiciones peores que las contenidas en los contratos, prevalecerán las condiciones más beneficiosas que se fijaron anteriormente. \\
		\hline
		\textbf{Principio de \textit{in dubio pro operario}} & Este principio significa que los tribunales, en caso de duda sobre la aplicación de la norma, la interpretarán de la forma que resulte más beneficiosa para el trabajador. \\
		\hline
	\end{longtable}
	
	\section*{El contrato}
	
	Un contrato es un negocio jurídico bilateral en el cual dos o más partes expresan su consentimiento en la forma permitida por la ley, para crear, regular, modificar o extinguir obligaciones.
	
	\subsection*{Elementos naturales del contrato}
	
	\begin{enumerate}
		\item Consentimiento de los contratantes. (capacidad y consentimiento)
		\item Objeto cierto que sea materia del contrato. (objeto)
		\item Causa de la obligación que se establezca. (causa)
	\end{enumerate}
	
	\subsection*{Elementos accidentales del contrato}
	
	Son aquellos que son adicionales al contrato, se pueden pactar o no, y no pueden ser contrarios a la ley, moral, las buenas costumbres, o el orden público.
	
	\section*{Tipos de contrato}
	
	\subsection*{Contrato civil}
	
	Es un acto jurídico donde dos o más partes de mutuo acuerdo crean una serie de obligaciones y a su vez generan derechos.
	
	\subsection*{Contrato mercantil}
	
	Es un contrato de presentación de servicios o entrega de bienes \textbf{cuyas partes son profesional o empresario} y colaboran de forma independiente.
	
	\subsection*{Contrato laboral}
	
	Es un acuerdo entre empresario y trabajador por el que éste se obliga a prestar determinados servicios por cuenta del empresario y bajo su dirección, a cambio de una retribución.
	
	\section*{Contrato indefinido ordinario}
	
	\begin{itemize}
		\item \textbf{Es aquel que se concierta sin establecer límites de tiempo} en la prestación de servicios, en cuanto a la duración del contrato.
		\item El contrato de trabajo indefinido podrá ser \textbf{verbal o escrito}.
		\item El contrato de trabajo indefinido podrá celebrarse \textbf{a jornada completa, parcial o para la prestación de servicios fijos discontinuos}.
		\item Pueden en algunos casos ser \textbf{beneficiarios de incentivos a la contratación}.
	\end{itemize}
	
	\subsection*{Formalización}
	
	El contenido del contrato se comunicará al SEPE en el plazo de los 10 días naturales siguientes a su concertación.
	
	\section*{Contrato indefinido adscrito a obra}
	
	Tendrán la consideración de contratos indefinidos adscritos a obra aquellos que tengan por objeto tareas o servicios cuya finalidad y resultado estén \textbf{vinculados a obras de la construcción, teniendo en cuenta las actividades establecidas en el ámbito funcional del Convenio General del Sector de la Construcción.}
	
	\subsection*{Características especiales de este tipo de contrato}
	
	\begin{itemize}
		\item Propuesta de recolocación
		\begin{itemize}
			\item Al finalizar la obra en la que se presta sus servicios el trabajador, la empresa debe efectuar una propuesta de recolocación, previo desarrollo, de ser preciso, de un proceso de formación a cargo de la empresa.
			\item La propuesta de recolocación \textbf{será formalizada por escrito}.
		\end{itemize}
		\item Proceso de formación para la recolocación
		\begin{itemize}
			\item La formación, esta será siempre a cargo de la empresa.
			\item La formación se impartirá \textbf{dentro de la jornada ordinaria} de las personas trabajadoras siempre que las circunstancias organizativas de la empresa lo permita.
			\item \textbf{Cuando estas circunstancias no lo permitan} se efectuará fuera de la jornada ordinaria pero el tiempo empleado en las horas efectivas de formación del curso \textbf{tendrá la consideración de tiempo de trabajo ordinario siendo retribuido} a valor de hora ordinario de la tabla del convenio aplicable \textbf{o compensado en tiempo de descanso.}
			\item El proceso de formación tendrá una duración de un máximo de 20 horas.
		\end{itemize}
		\item Finalización de obra
		\begin{itemize}
			\item Por finalización de las obras y servicios se entiende "la terminación real, verificable y efectiva de los trabajos desarrollados.
		\end{itemize}
		\item Extinción del contrato
		\begin{enumerate}
			\item La persona trabajadora afectada rechaza la recolocación.
			\item La cualificación de la persona afectada, incluso tras un proceso de formación o recualificación, no resulta adecuada a las nuevas obras que tenga la empresa en la misma provincia.
			\item La inexistencia en la provincia en la que esté contratada la persona trabajadora de obras.
		\end{enumerate}
		\item Consecuencias de la extinción
		\begin{itemize}
			\item La extinción del contrato implica el abono al trabajador de una indemnización del siete por ciento.
		\end{itemize}
	\end{itemize}
	
	\section*{Contrato fijo discontinuo}
	
	Este tipo de contrato es un contrato indefinido, aunque intermitente, es decir, se trabaja en un periodo determinado a lo largo del año y en el resto la relación laboral se mantiene suspendida.
	
	Está pensado para \textbf{desempeñar trabajo en una empresa que se repiten con una fecha incierta dentro de su actividad normal.}
	
	\subsection*{Llamamiento}
	
	\textbf{Los trabajadores serán llamados en el orden y la forma que se determine en el convenio colectivo.}
	
	En todo caso, \textbf{el llamamiento deberá realizarse por escrito} o por cualquier otro medio que permita que deje constancia de la fecha de llamamiento y de las condiciones laborales de esta siempre con una antelación adecuada.
	
	\textbf{Baja voluntaria:} Si el trabajador no acude al llamamiento de forma voluntaria, se consideraría una baja voluntaria del trabajador y, por tanto, finalizaría la relación laboral sin derecho a indemnización por despido ni a desempleo.
	
	\subsection*{Finiquito al finalizar el periodo de trabajo}
	
	\textbf{La relación laboral se debe finiquitar en cada periodo de trabajo}.
	
	Dicho finiquito contendrá las vacaciones generadas y no disfrutadas, así como la parte proporcional de las pagas extras, si es que no las tienes prorrateadas.
	
	Sin embargo, \textbf{no se debe abonar una indemnización}.
	
	\subsection*{Antigüedad en estos contratos}
	
	La persona trabajadoras fijas-discontinuas tienen derecho a que su antigüedad se calcule teniendo en cuenta toda la duración de la relación laboral y no el tiempo de servicios efectivamente prestados.
	
	No obstante, esta antigüedad no computa para los despidos, es decir, que solo se tendrá en cuenta los periodos efectivamente trabajados.
	
	\section*{Contrato temporal}
	
	Es aquel que tiene por objeto el establecimiento de una relación laboral entre la empresa y la persona trabajadora por un tiempo determinado.
	
	Para que se entienda que concurre causa justificada de temporalidad será necesario que se especifique:
	
	\begin{itemize}
		\item la causa habilitante de la contratación temporal.
		\item las circunstancias concretas que lo justifican
		\item conexión con la duración prevista.
	\end{itemize}
	
	\subsection*{Tipos}
	
	Se pueden dar como situaciones de temporalidad causadas por el objeto de la contratación:
	
	\begin{itemize}
		\item por circunstancias de la producción
		\item sustitución de persona trabajadora
		\item cobertura temporal de un puesto durante el proceso de selección o promoción.
	\end{itemize}
	
	\subsection*{Formalización}
	
	Constarán por escrito los contratos por tiempo determinado cuya duración sea superior a cuatro semanas.
	
	\subsection*{Cláusulas específicas del contrato por circunstancias de la producción}
	
	Se entenderá por circunstancias de la producción:
	
	\begin{enumerate}
		\item El incremento ocasional e imprevisible y las oscilaciones que, aun tratándose de la actividad normal de la empresa, generan un desajuste temporal entre el empleo estable disponible y el que se requiere.
		
		Su duración no podrá ser superior a seis meses. Por convenio colectivo de ámbito sectorial se podrá ampliar la duración máxima del contrato hasta un año.
		
		\item Igualmente, las empresas podrán formalizar contratos por circunstancias de la producción para atender situaciones ocasionales, previsibles y que tengan una duración reducida y delimitada en los términos previstos en este párrafo. Las empresas solo podrán utilizar este contrato un máximo de noventa días en el año natural, independientemente de las personas trabajadoras que sean necesarias.
		
		No podrán ser utilizados de manera continua.
	\end{enumerate}
	
	\subsubsection*{Características del contrato}
	
	Indemnización de doce días de salario por cada año de servicio.
	
	\subsection*{Cláusulas específicas del contrato temporal de sustitución de persona trabajadora}
	
	Sustituir a personas trabajadoras con derecho a reserva del puesto de trabajo, en virtud de norma, convenio colectivo o acuerdo individual, o para cubrir temporalmente un puesto de trabajo durante el proceso de selección o promoción para su cobertura definitiva.
	
	\subsubsection*{Características del contrato}
	
	Cuando el contrato se celebre para la sustitución de una persona trabajadora con derecho a reserva de puesto de trabajo, siempre que se especifique en el contrato el nombre de la persona sustituida y la causa de la sustitución. En tal supuesto, la prestación de servicios podrá iniciarse antes de que se produzca la ausencia de la persona sustituida, coincidiendo en el desarrollo de las funciones el tiempo imprescindible para garantizar el desempeño adecuado del puesto y, como máximo, durante quince días.
	
	\subsection*{Cláusulas específicas del contrato temporal de situación de jubilación parcial}
	
	Requisitos de las personas trabajadoras
	
	\begin{enumerate}
		\item La persona trabajadora que accede a la jubilación parcial debe tener al menos 60 años cumplidos.
		
		\item El acceso estará en función de los periodos cotizados:
		\begin{itemize}
			\item 33 años de cotizaciones efectivas en general.
			\item 25 años, en el supuesto de personas con discapacidad en grado igual o superior al 33 \%.
		\end{itemize}
		
		3. También se exige un periodo de antigüedad en la empresa de al menos, 6 años inmediatamente anteriores a la fecha de la jubilación parcial.
	\end{enumerate}
	
	\section*{Contrato para la formación en alternancia}
	
	De acuerdo con lo previsto en el artículo 11.2 del Estatuto de los Trabajadores, el contrato para la formación en alternancia tendrá por objeto compatibilizar la actividad laboral retribuida con los correspondientes procesos formativos en el ámbito de la formación profesional, los estudios universitarios o de certificados de profesionalidad.
	
	A tal fin, el puesto de trabajo debe permitir la formación complementaria prevista y la actividad laboral desempeñada en la empresa deberá estar directamente relacionada con la actividad formativa que justifica la contratación laboral.
	
	\subsection*{Duración}
	
	Solo podrá celebrarse un contrato de formación en alternancia por cada ciclo formativo de formación profesional y titulación universitaria, certificado de profesionalidad.
	
	No obstante, podrán formalizarse contratos de formación en alternancia con varias empresas.
	
	La duración del contrato será la prevista en el correspondiente plan o programa formativo, con un mínimo de tres meses y máximo de dos años.
	
	\subsection*{Requisitos de los trabajadores}
	
	Sin perjuicio de lo anterior, se podrán realizar con personas que posean otra titulación, siempre que no sea del mismo nivel formativo y del mismo sector productivo.
	
	\subsection*{Jornada de los contratos}
	
	El tiempo de trabajo efectivo no podrá ser superior al 65 \%, durante el primer año, o al 85 \%, durante el segundo, de la jornada máxima prevista en el convenio colectivo de aplicación en la empresa, o, en su defecto, de la jornada máxima legal.
	
	\subsection*{Retribución}
	
	No podrá ser inferior al 60 \%, durante el primer año, o al 85 \%, durante el segundo. En ningún caso la retribución podrá ser inferior al salario mínimo interprofesional en proporción al tiempo de trabajo efectivo.
	
	\subsection*{Actividad formativa}
	
	El contrato para la formación en alternancia deberá incorporar como anexo el convenio de colaboración.
	
	\subsection*{Acción protectora de la Seguridad Social}
	
	La acción protectora de la Seguridad Social para la formación en alternancia, comprenderá todas las contingencias, situaciones protegibles y prestaciones, incluido el desempleo. Asimismo, se tendrá derecho a la cobertura del Fondo de Garantía Salarial.
	
	\section*{Contrato formativo para la obtención de la práctica profesional}
	
	Tendrá por objeto la obtención de la práctica profesional adecuada al nivel de estudios o de formación objeto del contrato.
	
	\subsection*{Duración}
	
	La duración de este contrato no podrá ser inferior a seis meses ni exceder de un año.
	
	\subsection*{Requisitos de los trabajadores}
	
	Deberá concertarse dentro de los tres años, o de los cinco años si se concierta con una persona con discapacidad, siguientes a la terminación de los correspondientes estudios.
	
	Las personas que hayan hecho sus estudios en el extranjero, dicho cómputo se efectuará desde la fecha del reconocimiento u homologación del título en España.
	
	\subsection*{Periodo de prueba}
	
	Se podrá establecer un periodo de prueba que en ningún caso podrá exceder de un mes, salvo lo dispuesto en convenio colectivo.
	
	\subsection*{Actividad laboral desarrollada por la persona trabajadora}
	
	El puesto de trabajo deberá permitir la obtención de la práctica profesional adecuada al nivel de estudios o de formación objeto del contrato.
	
	\subsection*{Actividad formativa}
	
	A la finalización del contrato para la obtención de la práctica profesional, la persona trabajadora tendrá derecho a la certificación del contenido de la práctica realizada.
	
	\subsection*{Otros}
	
	\begin{itemize}
		\item Las empresas podrán solicitar por escrito al SEPE información relativa a si las personas a las que pretenden contratar han estado previamente contratadas bajo esta modalidad y la duración de estas contrataciones.
	\end{itemize}
	
	\section*{Modificación, suspensión y extinción del contrato de trabajo}
	
	\subsection*{Suspensión del contrato de trabajo}
	
	\subsubsection*{¿Qué es?}
	
	La interrupción temporal de la prestación laboral sin quedar roto el vínculo contractual entre la empresa y trabajador.
	
	\subsubsection*{Efectos}
	
	La suspensión del contrato deja sin efectos las obligaciones de ambas partes: trabajar y remunerar el trabajo. En algunos casos el trabajador percibirá una prestación de la Seguridad Social sustitutoria del salario.
	
	\subsubsection*{Reincorporación al trabajo y duración de la suspensión}
	
	\begin{enumerate}
		\item El trabajador tiene derecho a reincorporarse al trabajo que ocupaba una vez cesen las causas que motivaron la suspensión \textbf{excepto en los supuestos de suspensión por mutuo acuerdo} de las partes y por causas consignadas válidamente en el contrato, en que se estará a lo pactado.
		
		2. \textbf{Ejercicio de cargo público representativo o funciones sindicales} de ámbito provincial o superior (excedencia forzosa): Deberá \textbf{reincorporarse en} el plazo máximo de treinta \textbf{días naturales a partir del cese en el ejercicio de cargo}.
	\end{enumerate}
	
	\subsection*{Excedencias}
	
	Se entiende por estas las situaciones de suspensión del contrato de trabajo a solicitud del trabajador.
	
	\subsubsection*{Clases}
	
	\begin{itemize}
		\item Voluntaria
		\begin{itemize}
			\item Se requiere antigüedad de un año mínimo en la empresa.
			\item No se reconoce derecho a reserva del puesto de trabajo sino derecho preferente de reingreso cuando haya vacante de igual o similar categoría.
			\item Su duración será entre cuatro meses y cinco años.
			\item Este derecho solo podrá ser ejercido otra vez por el mismo trabajador si han transcurrido cuatro años desde el final de la anterior excedencia voluntaria.
		\end{itemize}
	\end{itemize}
	
	\subsection*{Extinción del contrato de trabajo (despido)}
	
	Significa la terminación de la relación laboral entre empresa y trabajador.
	
	El empresario, con ocasión de la extinción del contrato, al comunicar a los trabajadores la denuncia, o, en su caso, el preaviso de la extinción del mismo deberá acompañar una propuesta del documento de liquidación de las cantidades adecuadas.
	
\end{document}