\documentclass{article}
\usepackage[utf8]{inputenc}
\usepackage[spanish]{babel}
\usepackage{hyperref} % Para habilitar los enlaces
\usepackage{tocloft}  % Opcional: para personalizar el estilo del índice

\title{Trabajo de archivos}
\author{
Celeste \and Dario \and David
}
\date{\today}

\hypersetup{hidelinks}

\begin{document}

\maketitle

\tableofcontents

\section{Definición del fichero}

Un fichero es un conjunto ordenado de datos que tienen entre sí una relación lógica y están almacenados en un soporte de información adecuado.

En un mismo fichero se almacena información referente a un mismo tema de una forma estructurada con el fin de manipular los datos de forma individual.

\hrulefill

\section{Clases de archivo}

\subsection{Por su localización}

Los tipos de archivos físicos se pueden clasificar según su localización y la forma en que están organizados:

\subsubsection{Archivos de Archivo Central o Archivadores de Oficina}

\textbf{Localización:} Están ubicados en áreas comunes de la oficina o de trabajo.

\begin{itemize}
    \item   Archivadores de cajones: Son muebles con varios cajones donde se guardan documentos, y pueden ser verticales u horizontales.
    \item   Carpetas de archivo: Muebles o estantes donde se almacenan carpetas y documentos de manera ordenada.
\end{itemize}

\subsubsection{Archivos en Estantes o Libreros}

\textbf{Localización:} Usualmente en áreas de trabajo individuales o en estanterías fijas ubicadas en el espacio de oficina.

\begin{itemize}
    \item   Libreros metálicos o de madera: Son estantes abiertos o cerrados que permiten organizar documentos en carpetas o archivos.
    \item   Estantes modulares: Pueden personalizarse según las necesidades de almacenamiento.
\end{itemize}

\subsubsection{Archivos Móviles o Deslizantes}

\textbf{Localización:} En espacios más grandes o donde el acceso a los documentos debe ser rápido y fácil.

\subsubsection{Archivos en Áreas de Almacenamiento de Bajo Acceso}

\textbf{Localización:} En zonas menos accesibles de la oficina o en espacios de almacenamiento donde los documentos no se consultan con frecuencia.

\begin{itemize}
    \item   Estanterías altas: Usadas para almacenamiento de documentos que no requieren consulta frecuente.
    \item   Archivos en contenedores cerrados: Para documentos de archivo antiguo, generalmente etiquetados y guardados en cajas archivadoras.
\end{itemize}

\subsubsection{Archivos de Seguridad}

\textbf{Localización:} En espacios o áreas restringidas de la oficina o edificio, donde se requiere una protección adicional para la confidencialidad.

\begin{itemize}
    \item   Archivos de seguridad o archivadores con llave: Muebles que ofrecen un sistema de seguridad para documentos confidenciales.
    \item   Archivos ignífugos: Muebles diseñados para proteger los documentos en caso de incendio.
\end{itemize}

\subsubsection{Archivos en Áreas de Acceso Frecuente}

\textbf{Localización:} En escritorios o áreas donde los documentos se utilizan a menudo.

\textbf{Ejemplos de muebles:}

\begin{itemize}
    \item   Archivadores de sobremesa: Pequeños, ubicados en escritorios para tener los documentos cerca.
    \item   Cajas de archivo o bandejas organizadoras: Para mantener documentos a la mano y organizados.
\end{itemize}

\subsubsection{Archivos en Depósitos Externos o Almacenes}

\textbf{Localización:} En espacios de almacenamiento fuera de la oficina principal, como bodegas o depósitos.

\textbf{Ejemplos de muebles:}

\begin{itemize}
    \item   Estantes industriales o contenedores: Para almacenar grandes cantidades de documentos que no se necesitan de manera inmediata.
    \item   Archivos en cajas archivadoras: Para documentos de archivo que se guardan por largo tiempo.
\end{itemize}

\subsubsection{Archivos de Documentos Legales o Contables}

\textbf{Localización:} En áreas donde se requieren normativas específicas para almacenamiento de documentos legales o financieros.

\textbf{Ejemplos de muebles:}

\begin{itemize}
    \item   Archivos para documentos legales: Estantes o archivadores especializados para documentos que requieren conservación específica.
    \item   Cajas de archivo para documentos fiscales: Archivos que cumplen con regulaciones de conservación fiscal.
\end{itemize}

\subsection{Por su contenido}

\subsubsection{Documentos textuales:}

\begin{itemize}
    \item   Informativos: Aquellos cuyo contenido principal es la información escrita, como informes, memorandos, informes técnicos, artículos, ensayos, etc.
    \item   Normativos: Documentos que contienen normas, regulaciones, leyes o directrices, como leyes, reglamentos, códigos, estatutos, etc.
    \item   Narrativos: Documentos que relatan eventos o historias, como biografías, crónicas, novelas o relatos.
\end{itemize}

\subsubsection{Documentos gráficos}

\begin{itemize}
    \item   Cartográficos: Mapas y planos que representan información geográfica.
    \item   Gráficos: Diagramas, gráficos estadísticos, infografías, etc., que representan datos visualmente.
    \item   Fotográficos: Contienen imágenes, fotos o representaciones visuales de algún evento, objeto o persona.
\end{itemize}

\subsubsection{Documentos audiovisuales.}

\begin{itemize}
    \item   Sonoros: Archivos que contienen grabaciones de audio, como entrevistas, grabaciones históricas, podcasts, audios documentales, etc.
    \item   Video: Archivos con contenido en video que pueden ser películas, documentales, grabaciones históricas, reportajes, etc.
\end{itemize}

\subsubsection{Documentos mixtos:}

\begin{itemize}
    \item   Multimedia: Documentos que combinan diferentes tipos de contenido, como texto, imágenes, video y audio, en presentaciones, sitios web interactivos, o incluso documentos multimedia como CD, DVD, y archivos digitales en línea.
\end{itemize}

\subsubsection{Documentos de correspondencia:}

\begin{itemize}
    \item   Cartas y correos electrónicos: Son documentos de comunicación entre personas o entidades, tanto en formato físico (cartas) como digital (correos electrónicos).
\end{itemize}

\subsubsection{Documentos legales y administrativos:}

\begin{itemize}
    \item   Contratos y acuerdos: Documentos que contienen acuerdos formales entre partes, ya sean comerciales, laborales, etc.
    \item   Actas y resoluciones: Documentos legales que registran decisiones o acuerdos tomados en reuniones, tribunales o asambleas.
\end{itemize}

\subsubsection{Documentos contables y financieros:}

\begin{itemize}
    \item   Facturas, balances y estados financieros: Documentos que contienen información económica, contable o fiscal.
    \item   Recibos y contratos de pago: Documentos que evidencian transacciones financieras y compromisos de pago.
\end{itemize}

\subsection{Por la frecuencia de su uso}

\subsubsection{Documentos de uso frecuente (o activos):}

Son aquellos que se consultan o utilizan con regularidad en las actividades diarias de la organización o de la persona.

\begin{itemize}
    \item   Informes de trabajo o de progreso: Que se revisan periódicamente.
    \item   Correspondencia frecuente: Como correos electrónicos, memorandos y comunicaciones regulares.
    \item   Documentos operativos: Procedimientos, manuales de trabajo, formularios estándar, etc.
    \item   Contratos activos: Aquellos que están en vigor y son parte de la actividad cotidiana.
\end{itemize}

\subsubsection{Documentos de uso ocasional (o semiactivos):}

Son aquellos que no se utilizan todos los días, pero que se consultan con cierta periodicidad, generalmente en situaciones específicas o cuando se requieren para tareas puntuales.

\begin{itemize}
    \item   Actas de reuniones previas: Que pueden necesitarse de vez en cuando para hacer referencias históricas o aclaraciones.
    \item   Informes anuales: Que se consultan una vez al año para hacer análisis o auditorías.
    \item   Registros contables o fiscales: Que no se revisan de forma continua pero son importantes durante ciertos períodos fiscales.
\end{itemize}

\subsubsection{Documentos de uso esporádico (o inactivos):}

Son documentos que raramente se consultan, generalmente porque su contenido ya no tiene relevancia inmediata o están archivados por un largo periodo.

\begin{itemize}
    \item   Documentos históricos: Como archivos antiguos, registros que ya no son relevantes para las operaciones actuales.
    \item   Archivos de proyectos anteriores: Documentación de proyectos que ya han finalizado o se han cerrado y no requieren ser revisados.
    \item   Correspondencia obsoleta: Cartas o documentos que han perdido su vigencia y ya no tienen impacto en las actividades actuales.
\end{itemize}

\hrulefill

\section{Sistemas de Registro y Clasificación}

\begin{itemize}
    \item   Sistema alfabético
    \item   Sistema numérico
    \item   Sistema geográfico
    \item   Sistema cronológico
    \item   Sistema temático
\end{itemize}

\hrulefill

\section{Métodos de Mantenimiento de Archivos}

\begin{itemize}
    \item   Archivo horizontal
    \item   Archivo vertical
    \item   Archivo lateral
    \item   Digital
\end{itemize}

\hrulefill

\section{Unidades de Conservación de Archivos}

\begin{itemize}
    \item   Carpetas
    \item   Cajas
    \item   Legajos
    \item   Estanterías
    \item   Dispositivos digitales
\end{itemize}

\hrulefill

\section{Formas de Acceder al Archivo}

\begin{itemize}
    \item   Acceso físico
    \item   Acceso digital
\end{itemize}

\hrulefill

\section{Estándares de Retención de Documentos}

Son normas que establecen cuánto tiempo deben conservarse los diferentes tipos de documentos, según requisitos legales y organizacionales.


\hrulefill

\section{Definición de flujo y tipos de documentos}

\subsection{Flujo de documentos}

Es el proceso de cómo se distribuyen los documentos dentro de una organización, también se le conoce como flujo de trabajo de documentos.

\subsection{Flujo paralelo}

Es un proceso de trabajo que permite que varias tareas se realicen simultáneamente, en vez de seguir un orden secuencial.

\subsection{Flujo secuencial}

Es un proceso de trabajo que se basa en la ejecución de acciones en orden, una tras otra, se le conoce como flujo de trabajo secuencial.

\subsection{Flujo convergente}

Es un proceso que reúne varios paquetes de datos para formar uno más complejo.

\subsection{Flujo iterativo}

Este método permite resolver cómodamente una situación frecuente en todo tipo de algoritmos, es necesario ejecutar un conjunto de sentencias una y otra vez. Estas sentencias se repiten mientras determinada condición lógica se cumpla o un número determinado de veces.

\end{document}